\chapter[SGML and XML Parser for \FLORA]{SGML and XML Parsers for \FLORA\\ {by Rohan Shirwaikar}}



    This chapter documents the \FLORA package that provides SGML and
    XPath parsing capabilities. One set of predicates supports parsing
    SGML, XML, and HTML documents, which creates \FLORA objects in the user
    specified module. Other predicates evaluate XPath queries
    on XML documents and create \FLORA objects in user specified
    modules. The predicates make use of the {\bf sgml} and {\bf xpath}
    packages of XSB.
 


\section{Summary of the Predicates}

\begin{longtable}[l]{ll}
  {\bf \%load\_xml\_structure/3}&Parse XML data into \FLORA objects\\
  {\bf \%load\_sgml\_structure/3}&Parse SGML data into \FLORA objects\\
  {\bf \%load\_html\_structure/3}&Parse HTML data into \FLORA objects\\
  {\bf \%load\_xhtml\_structure/3}&Parse XHTML data into \FLORA objects\\
  {\bf \%parse\_xpath\_xml/3}&Apply an XPath expression to an XML
  document and parse the result\\
  {\bf \%parse\_xpath\_sgml/4}&Apply an XPath expression to an SGML
  document and parse the result\\
  {\bf \%parse\_xpath\_html/4}&Apply an XPath expression to an HTML document and parse the result\\
  {\bf \%parse\_xpath\_xhtml/4}&Apply an XPath expression to an XHTML document and parse the result\\
\end{longtable}

\section{Description}

This package supports parsing SGML, XML, and HTML documents,
converting them to sets of \FLORA objects stored in user-specified
\FLORA modules. The SGML interface
provides facilities to parse input in the form of files,
URLs and strings (Prolog atoms).  

For example, the following XML snippet

\begin{verbatim}
<greeting id='1'>
<first ssn=111'>
John
</first>
</greeting>
\end{verbatim}

will be converted into the following \FLORA objects:

\begin{verbatim}
o1[ element -> {o2}]
o2[ id = '1']
o2[ first -> {o3}]
o2[ ssn -> '111']
\end{verbatim}

To load the {\tt flrxml} package, the user should run the
following command at the \FLORA prompt.

%%
\begin{verbatim}
  flora?- [flrxml]. 
\end{verbatim}
%%

The following predicates are provided by the {\tt flrxml} package.  They
take SGML, XML, HTML, or XHTML documents and create the corresponding
\FLORA objects as specified in Section~\ref{xml-to-flora}.

\begin{description}
\item [\%load\_sgml\_structure({\tt +?Source,-?Warn,+?Module})@flrxml]
\item[\%load\_xml\_structure({\tt +?Source,-?Warn,+?Module})@flrxml]
\item[\%load\_html\_structure({\tt +?Source,-?Warn,+?Module})@flrxml]
\item[\%load\_xhtml\_structure({\tt +?Source,-?Warn,+?Module})@flrxml]
\end{description}
%%
The arguments to these predicates have the following meaning:

{\tt ?Source} is an input SGML, XML, HTML, or XHTML document.
It is of the form {\tt url({\tt {url}})},
{\tt file('{\tt {file name}}')} or {\tt string('{\tt document as a string}')}. 
{\tt ?Module} is the name of the \FLORA module where the objects created
by {\tt flrxml}  should be placed. {\tt ?Module}  must be bound.
{\tt ?Warn} gets bound to a list of warnings, if any are generated, or to
an empty list.
  

\section{XPath Support}

XPath support is based on the XSB {\tt xpath} package, which must be
configured as explained in the XSB manual. This package, in turn, relies on
the XML parser called {\tt libxml2}. It comes with most Linux
distributions and is also available for Windows, MacOS, and other
Unix-based systems from {\tt http://xmlsoft.org}. 
Note that both the library itself and the {\tt .h} files of that
library must be installed. 

The following predicates are provided. They select parts of the input
document using the provided XPath expression and create \FLORA objects as
specified in Section~\ref{xml-to-flora}. These predicates handle XML, SGML,
HTML, and XHTML, respectively.

\begin{description}
\item[\%parse\_xpath\_xml({\tt +?Source,+?XPath,+?NamespacePrefixList,-?Warn,+?Module})@flrxml]
\item[\%parse\_xpath\_sgml({\tt +?Source,+?XPath,+?NamespacePrefixList,-?Warn,+?Module})@flrxml]
\item[\%parse\_xpath\_html({\tt +?Source,+?XPath,+?NamespacePrefixList,-?Warn,+?Module})@flrxml]
\item[\%parse\_xpath\_xhtml({\tt +?Source,+?XPath,+?NamespacePrefixList,-?Warn,+?Module})@flrxml]
\end{description}
%%
The arguments have the following meaning:

{\tt Source} specifies the input document; this parameter has the same
format as in {\tt \%load\_structure}. {\tt ?XPath} is an XPath expression
specified as a Prolog atom. \emph{?Module} is the module where the resulting
\FLORA objects should be placed. {\tt ?Module} must be bound. \emph{?Warn}
gets bound to a list of warnings, if any are generated during the
processing, or to an empty list.

{\tt ?NamespacePrefixList} is a string that looks like a
space separated list of items of the form {\tt
  \emph{prefix} = \emph{namespaceURL}}. This allows one to use namespace
prefixes in the
XPath query given in the \emph{?XPath} parameter.
For example if the XPath expression is {\tt '/x:html/x:head/x:meta'}
where {\tt x} 
stands for {\tt 'http://www.w3.org/1999/xhtml'}, then this
prefix would have to be
defined in {\tt ?NamespacePrefixList}:

%%
\begin{verbatim}	
    parse_xpath_xhtml(url('http://w3.org'),
                      '/x:html/x:head/x:meta',
                      'x=http://www.w3.org/1999/xhtml',
                      ?Warnings,
                      foomodule)@flrxml.
\end{verbatim}
%%



\section{Mapping XML to \FLORA}\label{xml-to-flora}

This mapping is based on a proposal by Guizhen Yang.
It specifies how an XML parser can construct the
corresponding F-logic objects as a result of parsing an input XML
document.

The basic idea is as follows:
\begin{itemize}
\item Elements in XML are modeled as objects in F-logic.

\item Subelements in XML are modeled as multivalued attributes in F-logic.

\item Element attributes in XML are modeled as single-valued attributes
    in F-logic. This complies to the XML 1.0 specification which states
    that an attribute be defined only once for each element in an XML
    document.
\end{itemize}

This proposal deals with data-intensive XML documents, i.e., those that
don't rely on the interpretation of comments or processing instructions to
carry data. However, this proposal does consider the modeling of mixed
element content in which text and subelements are interspersed.

We do not consider modeling XML entities either, assuming no entity
references or that all entity references in the original XML document
are already resolved by an XML parser.



\subsection{Object Ids}

According to the XML specification 1.0, an XML element can be defined
with an oid that is unique across the document. Such an oid can be
provided as the value of an element attribute of type ID, although
this attribute can be arbitrarily named. Since an XML element is
modeled as an F-logic object, we would like the oid of this object to
take the value of any ID attribute if such value is defined.
Otherwise, the oid must be automatically generated by the system.

Sitting on top of the XML root element, there is an additional root
object which just functions as the access point to the entire object
hierarchy.

Note:
\begin{itemize}
\item Only a validating XML parser can decide whether an attribute
    is an ID attribute since such definition is provided by a DTD.

\item The oids of leaf nodes, which have no outgoing edges and carry
    plain text, are just the string values.
\end{itemize}

For example, the following XML document:

%%
\begin{verbatim}
<?xml version="1.0"?>
<person ssn="111-22-3333">
  <name first="John"                       (Example 1)
        last="Smith"/>
</person>
\end{verbatim}
%%
is represented by the following F-logic objects, provided we already
know that {\tt ssn}  is an ID attribute:
%%
\begin{verbatim}
   o1[person -> {'111-22-3333'}].
   '111-22-3333'[ssn -> '111-22-3333',
                 name -> {o2}
                ].
   o2[first -> 'John', last -> 'Smith'].
\end{verbatim}
%%


\subsection{Text and Mixed Element Content}

The content of an XML element may consist of plain text, or
subelements interspersed with plain text, for instance:

%%
\begin{verbatim}
  <greeting>Hi! My name is <first>John</first> <last>Smith</last>.</greeting> 
\end{verbatim}
%%

Each text segment is modeled in F-logic as if it were referred to by a
special tag, named {\tt '\$text'}.  Corresponding to each text segment, a node
is created and is referred to from the parent node by an edge labeled
{\tt '\$text'}. The text becomes the value of a special single-valued
attribute, named {\tt '\$string'},  of the newly created node. Moreover, if the
content of an XML element consists solely of plain text, then the text
also becomes the value of a special single-valued attribute of this
element, named {\tt '\$content'},  which is introduced for convenience purpose.

Therefore, the above XML segment would generate the following F-logic
objects, where {\tt o1, ..., o9}   are new oids:

\begin{verbatim}
o1[greeting -> {o2}].
o2['$text' -> {o3},
   first -> {o4},
   '$text' -> {o6},
   last  -> {o7},
   '$text' -> {o9}
  ].
o3['$string' -> 'Hi! My name is '].
o4['$content' -> 'John', '$text' -> {o5}].
o5['$string' -> 'John'].
o6['$string' -> ' '].
o7['$content' -> 'Smith', '$text' -> {o8}].
o8['$string' -> 'Smith'}.
o9['$string' -> '.'].
\end{verbatim}

Note:
\begin{itemize}
\item Handling of the whitespaces depends on the application. In the
  examples shown here, we assume that
  ``insignificant'' whitespaces have been omitted by the XML parser.
\item {\tt '\$content'}  can also be defined for every element. Its value is
    just the ASCII string as it appeared in the original document.
\end{itemize}

\subsection{Multivalued XML Attributes}

XML element attributes of type IDREFS are multivalued, in the sense
that their value is a string consisting of one or more oids separated
by whitespaces. Therefore, the value of such an attribute is a set. To
stick to the convention that element attributes are modeled as
single-valued attributes in F-logic, the value of an XML IDREFS
attribute is represented as a list.

For example, the following XML segment:
%%
\begin{verbatim}
<paper id="yk00" references="klw95 ckw91">
  <title>paper title</title>
</paper>
\end{verbatim}
%%
will generate the following F-logic atoms, assuming that the
{\tt reference}  attribute is of type IDREFS:
%%
\begin{verbatim}
yk00[references -> [klw95,ckw91],
       title -> {o1}
      ].
o1['$content' -> 'paper title', '$text' -> {o2}].
o2['$string' -> 'paper title'].
\end{verbatim}
%%
Note that since attribute definitions are provided by DTDs, only a
validating XML parser can decide whether an element attribute has the type
IDREFS.  A non-validating parser should just output the attribute values as
strings. The semantics of these strings is subject to further
interpretation by the applications.


\subsection{Ordering}

XML is order-sensitive, since XML DTDs impose order among elements.
For example, the following DTD element definition
%%
\begin{verbatim}
  <!ELEMENT book (author+, title, ISBN?)> 
\end{verbatim}
%%
states that the content of a {\tt book}  element consists of one or more
{\tt author}  element, followed by one {\tt title}  element, followed by an
optional {\tt ISBN}  element.
Note that XML 1.0 does not prescribe any order among element attributes.

Preserving the order of elements can also be useful for translating F-logic
objects back into an XML document.

A total order among the elements in an XML document should suffice to
establish the order in which the elements appear sequentially in the
XML document. In other words, such a total order should correspond to
the order in which the element open tags appear. In addition, the
ordering should also take into account the mixed element contents in
which each text segment is referred to by a {\tt '\$text'}  attribute.

\begin{verbatim}
<?xml version="1.0"?>
<person ssn="111-22-3333">
  <name>
    <first>John</first>                       (Example 2)
    <last>Smith</last>
  </name>
  <email>jsmith@abc.com</email>
</person>
\end{verbatim}


For example, the XML document in Example 2 can be represented by the
following tree, in which the integers enclosed by parentheses beside
the nodes represent the order assigned to the elements:

\begin{verbatim}
                    o1 (0)
                    |
                    | person
                    |
               '111-22-3333' (1)
                   / \
             name /   \ email
                 /     \
            (2) o2     o7 (7)
               / \       \ 
        first /   \last   \ '$text'
             /     \       \
        (3) o3     o5 (5)  o8 (8)
            |      |       'jsmith@abc.com'
    '$text' |      |'$text'
            |      |
        (4) o4     o6 (6)
           'John'  'Smith'
\end{verbatim}

Pre-order traversal of the XML tree will generate the total order for
the XML elements and text segments.

The ordering information that exists in XML documents is encoded as
follows. For each node, we introduce a special single-valued attribute,
named {\tt '\$order'}, which stores the order number.

\begin{verbatim}
<bibliography>
<paper id="sb97">
  <author>
    <first>John</first>
    <last>Smith</last>
  </author>
  <author>
    <first>David</first>
    <last>Brown</last>
  </author>                                  (Example 3)
</paper>

<paper id="s91">
  <author>
    <first>John</first>
    <last>Smith</last>
  </author>
</paper>
</bibliography>
\end{verbatim}


\subsection{More on Special Attributes}

The following special attributes are all single-valued.  Note that
this proposal does not intend to eliminate redundant information that
may exist across various attributes. It is up to the implementation to
decide whether to generate these attributes extensionally or
intentionally.

\begin{enumerate}
\item {\tt '\$tag'} \\
For each node, {\tt '\$tag'}  returns the unordered label of the edge
pointing to this node. {\tt '\$tag'}  can be defined as follows:
  %%
  \begin{verbatim}
     O['$tag' -> Tag] :- _[Tag -> {O}].
  \end{verbatim}
  %%
Note that for a node representing a text segment, the value of
its {\tt '\$tag'}  attribute is {\tt '\$text'}. 

\item {\tt '\$parent'} \\
For each node, {\tt '\$parent'}  returns the oid of the parent node.

\item {\tt '\$leftSibling'} \\
For each node, {\tt '\$leftSibling'}  returns the oid of the node appearing
immediately before the current node. This attribute is not defined
for the nodes without a left sibling.

\item {\tt '\$rightSibling'} \\
For each node, {\tt '\$rightSibling'}  returns the oid of the node appearing
immediately after the current node. This attribute is not defined
for those nodes without a right sibling.

\item {\tt '\$childrenNum'} \\
For each node, {\tt '\$childrenNum'}  returns the number of children including
nodes representing text segments.

\item {\tt '\$childrenList'} \\
For each node, {\tt '\$childrenList'}  returns a list, which is ordered, of the
oids of its children. Note that each text segment is also counted
as a child node.

\item {\tt '\$child'(N)} \\
For each node, {\tt '\$child'(N)} returns the {\tt N}-th child, where {\tt 1
  $\leq$ N} and {\tt N $\leq$ '\$chidlrenNum'}.

\item {\tt '\$tagList'} \\
For each node, {\tt '\$tagList'}  returns an ordered list of the
tags of its children. Note that each text segment is also counted
as if it were enclosed by a {\tt '\$text'}  tag.

\item {\tt '\$tag'(N)} \\
For each node, {\tt '\$tag'(N)}  returns the tag of the {\tt N}-th child, where {\tt 1
  $\leq$ N} and {\tt N $\leq$ '\$chidlrenNum'}.
This attribute can be defined as follows:

  %%
  \begin{verbatim}
     O['$tag'(N) -> Tag] :- O['$child'(N) -> V['$tag' -> Tag]]. 
  \end{verbatim}
  %%
\end{enumerate}

Note:
  The aforesaid attribute {\tt '\$content'}  can be defined
  for the nodes whose content is pure text as follows:

    %%
    \begin{verbatim}
      O['$content' -> String] :- 
	       O['$childrenNum' -> 1].text(1)['$string' -> String].
    \end{verbatim}
    %%

%%% Local Variables: 
%%% mode: latex
%%% TeX-master: "flora-packages"
%%% End: 
