\chapter[Persistent Modules]{Persistent Modules\\{by Vishal Chowdhary}}



\newcommand{\psm}{\mbox{PM}\xspace}

This chapter describes a \FLORA package that enables persistent modules.  A
\emph{persistent module} (abbr., \psm) is like any other \FLORA module
except that it is associated with a database. Any insertion or deletion of
base facts in such a module results in a corresponding operation on the
associated database. This data persists across \FLORA sessions, so the data
that was present in such a module is restored when the system restarts and
the module is reloaded.
  

\section{PM Interface}

A module becomes persistent by executing a statement that associates the
module with an ODBC data source. To start using the module persistence
feature, first load the following package into some module. For instance:
%% 
\begin{verbatim}
  ?- [persistentmodules>>pm].
\end{verbatim}
%% 
The following API is available. Note that if you load {\tt
  persistentmodules} into some other module, say {\tt foo}, then {\tt foo}
should be used instead of {\tt pm}.    
%%
\begin{itemize}
\item {\tt ?- ?Module[attach(?DSN,?DB,?User,?Password)]@pm.}\\
  This action associates the data source described by an ODBC {\tt DSN}
  with the module.  If {\tt ?DB} is a variable then the database is taken
  from the DSN. If {\tt ?DB} is bound to an atomic string, then that particular
  database is used. Not all DBMSs support the operation of replacing the
  DSN's database at run time. For instance, MS Access or PostgresSQL do not.
  In this case, {\tt ?DB} must stay unbound or else an error will be issued.
  For other DBMS, such as MySQL, SQL Server, and Oracle, {\tt ?DB} can be
  bound. 

  The {\tt ?User} and {\tt ?Password} must be bound to the user name and
  the password to be used to connect to the database.

  The database specified by the DSN must already exist. Otherwise, an
  exception of the form {\tt
    FLORA\_ABORT(FLORA\_DB\_EXCEPTION(?ErrorMsg,?))} is thrown.  (In a
  program, include {\tt flora\_exceptions.flh} to define {\tt
    FLORA\_DB\_EXCEPTION}; in the shell, use the symbol {\tt
    '\_\$flora\_db\_error'}.)  The database used in the {\tt attach}
  statement must not be accessed directly---only through the persistent
  modules interface.  The above statement will create the necessary tables
  in the database, if they are not already present.

  Note that the same database can be associated with several different
  modules. The package will not mix up the facts that belong to different
  modules.
\item {\tt ?- ?Module[attachNew(?DSN,?DB,?User,?Password)]@pm.}\\
  Like {\tt attach}, but a new database is created as specified by {\tt
    ?DSN}.  If the same database already exists, an exception of the form
  {\tt FLORA\_DB\_EXCEPTION(?ErrorMsg)} is thrown.  (In a program, include
  {\tt flora\_exceptions.flh} to define {\tt FLORA\_DB\_EXCEPTION}; in the
  shell, use the symbol {\tt '\_\$flora\_db\_error'}.)  This method creates
  all the necessary tables, if they are not already present.
  
  Note that this command works only with database systems that understand
  the SQL command {\tt CREATE DATABASE}. For instance, MS Access does not
  support this command and will cause an error.
\item {\tt ?- ?Module[detach]@pm.}\\
  Detaches the module from its database. The module is no longer persistent
  in the sense that subsequent changes are not reflected in any database.
  However, the earlier data is not lost. It stays in the database and the
  module can be reattached to that database.
\item {\tt ?- ?Module[loadDB]@pm.}\\
  On re-associating a module with a database (i.e., when {\tt
    ?Module[attach(?DSN, ?DB,?User,?Password)]@pm} is called in a new
  \FLORA session), database facts previously associated with the module are
  loaded back into it.  However, since the database may be large, \FLORA
  does not preload it into the main memory. Instead, facts are loaded
  on-demand.  If it is desired to have all these facts in main
  memory at once, the user can execute the above command. If no previous
  association between the module and a database is found, an exception is
  thrown.
\end{itemize}



Once a database is associated with the module, querying and insertion of
the data into the module is done as in the case of regular (transient)
modules.  Therefore \psm's provide a transparent and natural access to the
database and every query or update may, in principle, involve a database
operation.  For example, a query like {\tt ?- ?D[dept -> ped]@StonyBrook.}
may invoke the SQL {\tt SELECT} operation if module {\tt StonyBrook} is
associated with a database.  Similarly {\tt  insert\{a[b \fd
  c]@stonyBrook\}} and {\tt delete\{a[e \fd f]@stonyBrook\}}  will invoke
SQL {\tt INSERT} and {\tt DELETE} commands, respectively. Thus, \psm's provide
a high-level abstraction over the external database.

Note that if {\tt ?Module[loadDB]@pm} has been previously executed,
queries to a persistent module will \emph{not} access the database since
\FLORA will use its in-memory cache instead. However, insertion and
deletion of facts in such a module will still cause database operations.

\section{Examples}

Consider the following scenario sequence of operations.

%%
\begin{verbatim}
// Create new modules mod, db_mod1, db_mod2.
flora2 ?- newmodule{mod}, newmodule{db_mod1}, newmodule{db_mod2}.
flora2 ?- [persistentmodules>>pm].

// insert data into all three modules.
flora2 ?- insert{q(a)@mod,q(b)@mod,p(a,a)@mod}.
flora2 ?- insert{p(a,a)@db_mod1, p(a,b)@db_mod1}.
flora2 ?- insert{q(a)@db_mod2,q(b)@db_mod2,q(c)@db_mod2}.

//  Associate modules db_mod1, db_mod2 with an existing database db
//  The data source is described by the DSN mydb.
flora2 ?- db_mod1[attach(mydb,db,user,pwd)]@pm.
flora2 ?- db_mod2[attach(mydb,db,user,pwd)]@pm.

// insert more data into db_mod2 and mod.
flora2 ?- insert{a(p(a,b,c),d)@db_mod2}.
flora2 ?- insert{q(a)@mod,q(b)@mod,p(a,a)@mod}.

// shut down FLORA-2
flora2 ?- _halt.
\end{verbatim}
%%

\noindent
Restart the \FLORA system.

%%
\begin{verbatim}
// Create the same modules again
flora2 ?- newmodule{mod}, newmodule{db_mod1}, newmodule{db_mod2}.

// try to query the data in any of these modules.
flora2 ?- q(?X)@mod.
No.

flora2 ?- p(?X,?Y)@db_mod1.
No.

//  Attach the earlier database to db_mod1.
flora2 ?- [persistentmodules>>pm].
flora2 ?- db_mod1[attach(mydb,db,user,pwd)]@pm.

// try querying again...

// Module mod is still not associated with any database and nothing was
// inserted there even transiently, we have:
flora2 ?- q(?X)@mod.
No.

// But the following query retrieves data from the database associated
// with db_mod1.
flora2 ?- p(?X,?Y)@db_mod1.
?X = a,
?Y = a.

?X = a,
?Y = b.

Yes.

// Since db_mod2 was not re-attached to its database,
// it still has no data, and the query fails.
flora2 ?- q(?X)@db_mod2.

No.
\end{verbatim}
%%



%%% Local Variables: 
%%% mode: latex
%%% TeX-master: "flora-packages"
%%% End: 
