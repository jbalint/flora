

\chapter[JAVA Interface for \FLSYSTEM]{JAVA Interface for \FLSYSTEM\\by Aditi
      Pandit and Michael Kifer}


     This chapter documents the API for accessing \FLSYSTEM from a Java program.
  The API has two versions: a \emph{low-level API},
  which enables Java programs to
  send arbitrary queries to \FLSYSTEM and get results, and a \emph{high-level
    API}, which is more limited, but is easier to use. The high-level API
  establishes a correspondence between Java classes and \FLSYSTEM classes,
  which enables manipulation of \FLSYSTEM classes by executing appropriate
  methods on the corresponding Java classes. Both interfaces rely on the
  Java-XSB interface, called \emph{Interprolog}, provided by Declarativa.com.


\section{The Low-level Interface}
 The low-level API enables Java programs to send arbitrary queries
to \FLSYSTEM and get results. 

In order to be able to access \FLSYSTEM, the Java program must first establish
a session for a running instance of \FLSYSTEM. Multiple sessions can be active
at the same time. The knowledge bases in the different running instances
are completely independent. Sessions are instances of
the class {\tt
javaAPI.src.FloraSession}. This class provides methods
for opening/closing sessions and loading \FLSYSTEM knowledge bases
(which are also used in the high-level
interface). In addition, a session provides 
methods for executing arbitrary \FLSYSTEM queries. The following is the complete
list of the methods that are available in that class.
\begin{itemize}
\item
\begin{verbatim}
public FloraSession()
\end{verbatim}
    This method creates a connection to an instance of \FLSYSTEM.
The initialization parameters needed are:

{\tt JAVA\_BIN}. This variable points to the directory containing
the javac and java executable programs. This variable is specified in
the {\tt windowsVariables.bat} and  {\tt unixVariables.sh}  files in the
{\tt java} subdirectory of the \FLSYSTEM distribution.

{\tt XSB\_ROOT\_DIRECTORY}. This variable points to the directory
containing the XSB executable. This variable is specified in the {\tt
  flora\_settings.bat} and {\tt flora\_settings.sh} files in the {\tt java}
directory.  {\tt FLORA\_DIRECTORY}. This variable points to the directory
containing the \FLSYSTEM installation. This variable is specified in the {\tt
  flora\_settings.bat} and {\tt flora\_settings.sh} files in the {\tt java}
directory.  Both of these files are generated automatically by the {\tt
  makeflora} script and should be of no concern to the user under normal
circumstances.
\item {\tt close()} \\
  This method must be called to terminate a \FLSYSTEM session. Note that this does
  not terminate the Java program that initiated the session:
  to exit the Java program that talks to \FLSYSTEM, one needs to execute
  the statement
  %%
\begin{verbatim}
 System.exit();  
\end{verbatim}
  %%
  Note that just returning from the {\tt main} method is not enough. 

\item
\begin{verbatim}
public Iterator<FloraObject> ExecuteQuery(String command)
\end{verbatim}
    This method executes the \FLSYSTEM command given by the
parameter {\tt command}.  It is used to execute \FLSYSTEM queries that
do not require variable bindings to be returned back to Java or queries that
have only
a single variable to be returned. Each binding is represented as
an instance of the class {\tt javaAPI.src.FloraObject}.
The examples below illustrate how to process the results returned by this
method.

\item
\begin{verbatim}
public Iterator<HashMap<String,FloraObject>> ExecuteQuery(String query, Vector vars)
\end{verbatim}
  This method executes the \FLSYSTEM query given by the first argument. The
  Vector {\tt vars} (of strings) specifies the names of all the variables
  in the query for which bindings need to be returned. These variables are
  added to the vector using the method {\tt add} before calling
  {\tt ExecuteQuery}. For instance, {\tt vars.add("?X")}.  
  
  This version of {\tt ExecuteQuery} returns an iterator over all bindings
  returned by the \FLSYSTEM query.  Each binding is represented by a {\tt
    HashMap<String,FloraObject>} 
  object which can be used to obtain the value of each variable in the
  query (using the {\tt get()} method). The value of each variable returned
  is an instance of {\tt
    javaAPI.src.FloraObject}.

  See the examples below for how to handle the results
  returned by this method.

\item
\begin{verbatim}
void loadFile(String fileName,String moduleName)
\end{verbatim}
  This method loads the \FLSYSTEM program, specified by the parameter {\tt
    fileName} into the \FLSYSTEM module specified in {\tt moduleName}.
\item
\begin{verbatim}
void compileFile(String fileName,String moduleName)
\end{verbatim}
  This method compiles (but does not load)
  the \FLSYSTEM program, specified by the parameter {\tt
    fileName} for the \FLSYSTEM module specified in {\tt moduleName}.
\item
\begin{verbatim}
void addFile(String fileName,String moduleName)
\end{verbatim}
  This method adds the \FLSYSTEM program, specified by the parameter {\tt
    fileName} to an existing \FLSYSTEM module specified in {\tt moduleName}.
\item
\begin{verbatim}
void compileaddFile(String fileName,String moduleName)
\end{verbatim}
  This method compiles the \FLSYSTEM program, specified by the parameter {\tt
    fileName} for addition to the \FLSYSTEM module specified in {\tt moduleName}.
\end{itemize}

The code snippet below illustrates the low-level API.

\paragraph{Step 1: Writing a \FLSYSTEM program.}
  Let us assume that we have a file, called  {\tt flogic\_basics.flr},
  which contains the following information:
\begin{quote}
\begin{verbatim}
person :: object.
dangerous_hobby :: object.
john:employee.
employee::person.

bob:person.
tim:person.
betty:employee.

person[|age=>integer,
       kids=>person,
       salary(year)=>value,
       hobbies=>hobby,
       believes_in=>something,
       instances => person
|].

mary:employee[
    age->29,
    kids -> {tim,leo,betty},
    salary(1998) -> a_lot
].

tim[hobbies -> {stamps, snowboard}].
betty[hobbies->{fishing,diving}].

snowboard:dangerous_hobby.
diving:dangerous_hobby.

?_X[self-> ?_X].

person[|believes_in -> {something, something_else}|].
\end{verbatim}
\end{quote}

\paragraph{Step 2:  Writing a JAVA application to interface with \FLSYSTEM.}
 The following code loads a \FLSYSTEM program from a file and then passes
 queries to the knowledge base.

\begin{verbatim}
import java.util.*;
import net.sf.flora2.API.*;
import net.sf.flora2.API.util.*;

public class flogicbasicsExample {

    public static void main(String[] args) {
        // create a new session for a running instance of the engine
        FloraSession session = new FloraSession();
        System.out.println("Engine session started");

        // Assume that Java was called with -DINPUT_FILE=the-file-name
        String fileName = System.getProperty("INPUT_FILE");
        if(fileName == null || fileName.trim().length() == 0) {
            System.out.println("Invalid path to example file!");
            System.exit(0);

        }
        // load the program into module basic_mod
        session.loadFile(fileName,"basic_mod");

        /* Running queries from flogic_basics.flr */

        /* Query for persons */
        String command = "?X:person@basic_mod.";
        System.out.println("Query:"+command);
        Iterator<FloraObject> personObjs = session.ExecuteQuery(command);

        /* Printing out the person names and information about their kids */
        while (personObjs.hasNext()) {
            FloraObject personObj = personObjs.next();
            System.out.println("Person name:"+personObj);


        command = "person[instances -> ?X]@basic_mod.";
        System.out.println("Query:"+command);
        personObjs = session.ExecuteQuery(command);

        /* Prining out the person names  */
        while (personObjs.hasNext()) {
            Object personObj = personObjs.next();
            System.out.println("Person Id: "+personObj);
        }

        /* Example of ExecuteQuery with two arguments */
        Vector<String> vars = new Vector<String>();
        vars.add("?X");
        vars.add("?Y");

        Iterator<HashMap<String,FloraObject>> allmatches =
            session.ExecuteQuery("?X[believes_in -> ?Y]@basic_mod.",vars);
        System.out.println("Query:?X[believes_in -> ?Y]@basic_mod.");
        while(allmatches.hasNext()) {
            HashMap<String,FloraObject> firstmatch = allmatches.next();
            Object Xobj = firstmatch.get("?X");
            Object Yobj = firstmatch.get("?Y");
            System.out.println(Xobj+" believes in: "+?Yobj);
        }
        // quit the system
        session.close();
        System.exit(0);
    }
}
\end{verbatim}

\section{The High-Level Interface}

The high-level API operates by creating proxy Java classes for 
\FLSYSTEM classes selected by the user.
This enables the Java program to operate on \FLSYSTEM classes by
executing appropriate methods on the corresponding proxy Java classes.
The use of the high-level API involves a number of steps, as described below.

\textbf{Note}: This interface will not work for \FLSYSTEM programs that
use \emph{non-alphanumeric} names for methods and predicates. For instance,
if a program involves statements like \texttt{foo['bar\$\#123'->456]} then
the interface might generate syntactically incorrect Java proxy classes and
errors will be issued during the compilation. 

\paragraph{Stage 1: Writing a \FLSYSTEM file.}
We assume the same {\tt flogic\_basics.flr} file as in the previous
example.

\paragraph{Stage 2: Generating Java classes that serve as proxies for \FLSYSTEM classes.}
The \FLSYSTEM side of the Java-to-\FLSYSTEM high level API provides a predicate
to generate Java proxy classes for each \fl class which have a signature
declaration in the \FLSYSTEM knowledge base. A proxy class gets defined so
that it would have methods to manipulate the attributes and methods of the
corresponding \fl class for which signature declarations are available.  If
an \fl class has a declared value-returning attribute {\tt foobar} then the
proxy class will have the following methods. Each method name has the form
\emph{action}$S_1S_2S_3$\_{\tt foobar}, where \emph{action} is either {\tt
  get}, {\tt set}, or {\tt delete}. The specifier $S_1$ indicates the type
of the method --- {\tt V} for value-returning, {\tt B} for Boolean, and
{\tt P} for procedural. The specifier $S_2$ tells whether the operation
applies to the signature of the method ({\tt S}), e.g., {\tt
  person[foobar=>string]}, or to the actual data ({\tt D}), for example,
{\tt john[foobar->3]}.  Finally, the specifier $S_3$ tells if the operation
applies to the inheritable variant of the method ({\tt I})
or its non-inheritable variant ({\tt N}).
%% 
\begin{enumerate}
\item {\tt public Iterator<FloraObject> getVDI\_foobar()}\\
  {\tt public Iterator<FloraObject> getVDN\_foobar()}
  \\
  {\tt public Iterator<FloraObject> getVSI\_foobar()}\\
  {\tt public Iterator<FloraObject> getVSN\_foobar()}
  \\
  The above methods query the knowledge base and get all answers for the
  attribute {\tt foobar}. They return iterators through which these answers
  can be processed one-by-one. Each object returned by the iterator is of
  type {\tt FloraObject}.  The {\tt getVDN} form queries non-inheritable
  data methods and {\tt getVDI} the inheritable ones. The {\tt getVSI} and
  {\tt getVSN} forms query the signatures of the attribute {\tt foobar}.
\item {\tt public boolean setVDI\_foobar(Vector value)}\\
  {\tt public boolean setVDN\_foobar(Vector value)}
  \\
  {\tt public boolean setVSI\_foobar(Vector value)}\\
  {\tt public boolean setVSN\_foobar(Vector value)}
  \\
  These methods
  add values to the set of values returned by the attribute {\tt foobar}. The
  values must be placed in the vector parameter passed these methods.
  Again, {\tt setVDN} adds data for non-inheritable methods and {\tt setVDI}
  is used for inheritable methods.
  {\tt setVSI} and {\tt setVSN} add types to signatures.  
\item {\tt public boolean setVDI\_foobar(Object value)}\\
  {\tt public boolean setVDN\_foobar(Object value)}
  \\
  {\tt public boolean setVSI\_foobar(Object value)}\\
  {\tt public boolean setVSN\_foobar(Object value)}\\
  These methods provide a simplified interface when only one value needs to
  be added.  It works like the earlier set\_* methods, except that only one
  value given as an argument is added.
\item {\tt public boolean deleteVDI\_foobar(Vector value)}\\
  {\tt public boolean deleteVDN\_foobar(Vector value)}
  \\
  {\tt public boolean deleteVSI\_foobar(Vector value)}\\
  {\tt public boolean deleteVSN\_foobar(Vector value)}
  \\
  Delete a set of values of the attribute {\tt foobar}. The set is
  specified in the vector argument.
\item {\tt public boolean deleteVDI\_foobar(Object value)}\\
  {\tt public boolean deleteVDN\_foobar(Object value)}
  \\
  {\tt public boolean deleteVSI\_foobar(Object value)}\\
  {\tt public boolean deleteVSN\_foobar(Object value)}
  \\
  A simplified interface for the case when only one value needs to be deleted.
\item {\tt public boolean deleteVDI\_foobar()}\\
  {\tt public boolean deleteVDN\_foobar()}
  \\
  {\tt public boolean deleteVSI\_foobar()}\\
  {\tt public boolean deleteVSN\_foobar()}
  \\
  Delete all values for the attribute {\tt foobar}. 
\end{enumerate}
%% 
For \fl methods with arguments, the high-level API provides Java methods as
above, but they take more arguments to accommodate the parameters that \fl
methods take. Let us
assume that the \fl method is called {\tt foobar2} and it takes parameters
{\tt arg1} and {\tt arg2}.  As before the {\tt getVDI\_*}, {\tt setVDI\_*},
etc., forms of the Java methods are for dealing with inheritable \FLSYSTEM
methods and the {\tt getVDN\_*}, {\tt setVDN\_*},
etc., forms are for dealing with non-inheritable \FLSYSTEM methods.

%%
\begin{enumerate}
\item {\tt public Iterator<FloraObject> getVDI\_foobar2(Object arg1, Object arg2)}\\
  {\tt public Iterator<FloraObject> getVDN\_foobar2(Object arg1, Object arg2)}
  \\
  Obtain all values for the \fl method invocation {\tt foobar2(arg1,arg2)}.
\item {\tt public boolean setVDI\_foobar2(Object arg1, Object arg2, Vector value)}\\
  {\tt public boolean setVDN\_foobar2(Object arg1, Object arg2, Vector value)}
  \\
  Add a set of methods specified in the parameter {\tt value} for the method
  invocation {\tt foobar2(arg1,arg2)}. 
\item {\tt public boolean setVDI\_foobar2(Object arg1, Object arg2, Object value)}\\
  {\tt public boolean setVDN\_foobar2(Object arg1, Object arg2, Object value)}
  \\
  A simplified interface when only one value is to be added.
\item {\tt public boolean deleteVDI\_foobar2(Object arg1, Object arg2, Vector value)}\\
  {\tt public boolean deleteVDN\_foobar2(Object arg1, Object arg2, Vector value)}
  \\
  Delete a set of values from {\tt foobar2(arg1,arg2)}. The set is given by
  the vector parameter {\tt value}. 
\item {\tt public boolean deleteVDI\_foobar2(Object arg1, Object arg2, Object value)}\\
  {\tt public boolean deleteVDN\_foobar2(Object arg1, Object arg2, Object value)}
  \\
  A simplified interface for deleting a single value.
\item {\tt public boolean deleteVDI\_foobar2(Object arg1, Object arg2)}\\
  {\tt public boolean deleteVDN\_foobar2(Object arg1, Object arg2)}
  \\
  Delete all values for the method invocation {\tt foobar2(arg1,arg2)}. 
\end{enumerate}
%% 
For Boolean and procedural methods, the generated methods are similar
except that there is only one version for the set and delete methods. In
addition, Boolean inheritable methods use the {\tt getBDI\_*}, {\tt
  setBDI\_*}, etc., form, while non-inheritable methods use the {\tt
  getBDN\_*}, etc., form.  Procedural methods use the {\tt getPDI\_*}, {\tt
  getPDN\_*}, etc., forms.  For instance,
%% 
\begin{enumerate}
\item  {\tt public boolean getBDI\_foobar3()}   \\
  {\tt public boolean getBDN\_foobar3()} \\
  {\tt public boolean getPDI\_foobar3()}   \\
  {\tt public boolean getPDN\_foobar3()}
\item {\tt public boolean setBDI\_foobar3()}   \\
  {\tt public boolean setBDN\_foobar3()} \\
  {\tt public boolean setPDI\_foobar3()}   \\
  {\tt public boolean setPDN\_foobar3()}
\item {\tt public boolean deleteBDI\_foobar3()}  \\
  {\tt public boolean deleteBDN\_foobar3()}  \\
  {\tt public boolean deletePDI\_foobar3()}  \\
  {\tt public boolean deletePDN\_foobar3()}  
\end{enumerate}
%% 

In addition, the methods to query the ISA hierarchy are available:
%% 
\begin{itemize}
\item  {\tt public Iterator<FloraObject> getDirectInstances()}
\item  {\tt public Iterator<FloraObject> getInstances()}
\item  {\tt public Iterator<FloraObject> getDirectSubClasses()}
\item  {\tt public Iterator<FloraObject> getSubClasses()}
\item   {\tt public Iterator<FloraObject> getSuperClasses()}   
\item   {\tt public Iterator<FloraObject> getDirectSuperClasses()}   
\end{itemize}
%% 
These methods apply to the java proxy object that corresponds to the \fl
class person.

All these methods are generated automatically by executing the following
\FLSYSTEM query (defined in the \texttt{javaAPI} package).
All arguments in the query must be bound:
%% 
\begin{verbatim}
 // write(?Class,?Module,?ProxyClassFileName).
 ?- write(foo,example,'myproject/foo.java').
\end{verbatim}
%% 
The first argument specifies the class for which to generate the methods,
the file name tells where to put the Java file for the proxy object,
and the model argument tells which \FLSYSTEM model to load this program to. The
result of this execution will be the file {\tt foo.java} which should be
included with your java program (the program that is going to interface with
\FLSYSTEM). Note that because of the Java conventions, the file name must have
the same name as the class name.
It is important to remember, however, that proxy methods will
be generated only for those \fl methods that have been declared using
signatures.

Let us now come back to our program {\tt flogic\_basics.flr} for which we
want to use the high-level API.  Suppose we want to query the person class.
To generate the proxy declarations for that class, we create
the file {\tt person.java} for the 
module {\tt basic\_mod} as follows.
%%
\begin{quote}
\begin{verbatim}
?- load{'examples/flogic_basics'>>basic_mod}.
?- load{javaAPI}.
?- write(person,basic_mod,'examples/person.java')@\prolog
\end{verbatim}
\end{quote}


The {\tt write} method will create the file {\tt person.java} shown
below.  The methods defined in {\tt person.java} are the class constructors
for {\tt person}, the methods to query the ISA hierarchy, and the ``get'',
``set'' and ``delete'' methods for each method and attribute declared in
the \FLSYSTEM class {\tt person}.  The parameters for the ``get'', ``set'' and
``delete'' Java methods are the same as for the corresponding \FLSYSTEM
methods. The first constructor for class {\tt person} takes a low-level
object of class {\tt javaAPI.src.FloraObject} as a
parameter. The second parameter is the \FLSYSTEM module for which the proxy
object is to be created.
The second {\tt person}-constructor takes \fl object Id instead of a
low-level {\tt FloraObject}. It also takes the module name, as before, but,
in addition, it takes a session for a running \FLSYSTEM instance.
The session parameter was not needed for the first {\tt person}-constructor
because {\tt FloraObject} is already attached to a concrete session.  

It can be seen from the form of the proxy object constructors that
proxy objects are attached to specific \FLSYSTEM modules, which may seem to
go against the general philosophy that \fl objects do not belong to any
module --- only their methods do. On closer examination, however, attaching
high-level proxy Java objects to modules makes perfect sense. Indeed, a
proxy object encapsulates operations for manipulating \fl attributes 
and methods, which belong to concrete \FLSYSTEM modules, so the proxy object
needs to know which module it operates upon.


\underline{{\bf person.java file}}

\begin{quote}
\begin{verbatim}
import java.util.*;
import net.sf.flora2.API.*;
import net.sf.flora2.API.util.*;

public class person {

    public FloraObject sourceFloraObject;

    // proxy objects' constructors
    public person(FloraObject sourceFloraObject, String moduleName) { ... }
    public person(String floraOID,String moduleName, FloraSession session) { ... }

    // ISA hierarchy queries
    public Iterator<FloraObject> getDirectInstances() { ... }
    public Iterator<FloraObject> getInstances() { ... }
    public Iterator<FloraObject> getDirectSubClasses() { ... }
    public Iterator<FloraObject> getSubClasses() { ... }
    public Iterator<FloraObject> getDirectSuperClasses() { ... }
    public Iterator<FloraObject> getSuperClasses() { ... }

    // Java methods for manipulating methods
    public boolean setVDI_age(Object value) { ... }
    public boolean setVDN_age(Object value) { ... }
    public Iterator<FloraObject> getVDI_age(){ ... }
    public Iterator<FloraObject> getVDN_age(){ ... }
    public boolean deleteVDI_age(Object value) { ... }
    public boolean deleteVDN_age(Object value) { ... }
    public boolean deleteVDI_age() { ... }
    public boolean deleteVDN_age() { ... }
    public boolean setVDI_salary(Object year,Object value) { ... }
    public boolean setVDN_salary(Object year,Object value) { ... }
    public Iterator<FloraObject> getVDI_salary(Object year) { ... }
    public Iterator<FloraObject> getVDN_salary(Object year) { ... }
    public boolean deleteVDI_salary(Object year,Object value) { ... }
    public boolean deleteVDN_salary(Object year,Object value) { ... }
    public boolean deleteVDI_salary(Object year) { ... }
    public boolean deleteVDN_salary(Object year) { ... }
    public boolean setVDI_hobbies(Vector value) { ... }
    public boolean setVDN_hobbies(Vector value) { ... }
    public Iterator<FloraObject> getVDI_hobbies(){ ... }
    public Iterator<FloraObject> getVDN_hobbies(){ ... }
    public boolean deleteVDI_hobbies(Vector value) { ... }
    public boolean deleteVDN_hobbies(Vector value) { ... }
    public boolean deleteVDI_hobbies(){ ... }
    public boolean deleteVDN_hobbies(){ ... }
    public boolean setVDI_instances(Vector value) { ... }
    public boolean setVDN_instances(Vector value) { ... }
    public Iterator<FloraObject> getVDI_instances(){ ... }
    public Iterator<FloraObject> getVDN_instances(){ ... }
    public boolean deleteVDI_instances(Vector value) { ... }
    public boolean deleteVDN_instances(Vector value) { ... }
    public boolean deleteVDI_instances(){ ... }
    public boolean deleteVDN_instances(){ ... }
    public boolean setVDI_kids(Vector value) { ... }
    public boolean setVDN_kids(Vector value) { ... }
    public Iterator<FloraObject> getVDI_kids(){ ... }
    public Iterator<FloraObject> getVDN_kids(){ ... }
    public boolean deleteVDI_kids(Vector value) { ... }
    public boolean deleteVDN_kids(Vector value) { ... }
    public boolean deleteVDI_kids(){ ... }
    public boolean deleteVDN_kids(){ ... }
    public boolean setVDI_believes_in(Vector value) { ... }
    public boolean setVDN_believes_in(Vector value) { ... }
    public Iterator<FloraObject> getVDI_believes_in(){ ... }
    public Iterator<FloraObject> getVDN_believes_in(){ ... }
    public boolean deleteVDI_believes_in(Vector value) { ... }
    public boolean deleteVDN_believes_in(Vector value) { ... }
    public boolean deleteVDI_believes_in(){ ... }
    public boolean deleteVDN_believes_in(){ ... }
}
\end{verbatim}
\end{quote}

\paragraph{Stage 3: Writing Java applications that use the high-level API.}

The following program ({\tt flogicbasicsExample.java}) shows several
queries that use the high-level interface. The
class {\tt person.java} is generated at the previous stage.
The methods of the high-level interface operate on Java objects that are
proxies for \FLSYSTEM objects. These Java objects are members of the class
{\tt javaAPI.src.FloraObject}.
Therefore, before one can use the high-level methods one need to first
retrieve the appropriate proxy objects on which to operate. This is done
by sending an appropriate query through the method {\tt ExecuteQuery}---the
same method that was used in the low-level interface.
Alternatively, {\tt person}-objects could be constructed using the
3-argument proxy constructor, which takes \fl oids.


\begin{quote}
\begin{verbatim}
import java.util.*;
import net.sf.flora2.API.*;
import net.sf.flora2.API.util.*;

public class flogicbasicsExample {

    public static void main(String[] args) {
        /* Initializing the session */
        FloraSession session = new FloraSession();
        System.out.println("Flora session started");

        String fileName = "examples/flogic_basics"; // must be a valid path
        /* Loading the flora file */
        session.loadFile(fileName,"basic_mod");

        // Retrieving instances of the class person through low-level API
        String command = "?X:person@basic_mod.";
        System.out.println("Query:"+command);
        Iterator<FloraObject> personObjs = session.ExecuteQuery(command);

        /* Print out person names and information about their kids */
        person currPerson = null;
        while (personObjs.hasNext()) {
            FloraObject personObj = personObjs.next();
            // Elevate personObj to the higher-level person-object
            currPerson =new person(personObj,"basic_mod");

            /* Set that person's age to 50 */
            currPerson.setVDN_age("50");

            /* Get this person's kids */
            Iterator<FloraObject> kidsItr = currPerson.getVDN_kids();
            while (kidsItr.hasNext()) {
                FloraObject kidObj = kidsItr.next();
                System.out.println("Person: " + personObj + " has kid: " +kidObj);

                person kidPerson = null;
                // Elevate kidObj to kidPerson
                kidPerson = new person(kidObj,"basic_mod");

                /* Get kidPerson's hobbies */
                Iterator<FloraObject> hobbiesItr = kidPerson.getVDN_hobbies();
                while(hobbiesItr.hasNext()) {
                    FloraObject hobbyObj = hobbiesItr.next();
                    System.out.println("Kid:"+kidObj + " has hobby:" +hobbyObj);
                }
            }
        }

        FloraObject age;
        // create a person-object directly by supplying its F-logic OID
        // father(mary)
        currPerson = new person("father(mary)", "example", session);
        Iterator<FloraObject> maryfatherItr = currPerson.getVDN_age();
        age = maryfatherItr.next();
        System.out.println("Mary's father is " + age + " years old");

        // create a proxy object for the F-logic class person itself
        person personClass = new person("person", "example", session);
        // query its instances through the high-level interface
        Iterator<FloraObject> instanceIter = personClass.getInstances();
        System.out.println("Person instances using high-level API:");
        while (instanceIter.hasNext())
            System.out.println("    " + instanceIter.next());
        
        session.close();
        System.exit();
    }
}
\end{verbatim}
\end{quote}

\paragraph{Stage 4: Running the Java application program.}
To run Java programs that interface with \FLSYSTEM, one must follow the
following guidelines.

\begin{itemize}
\item Place the files {\tt flogicsbasicsExample.java} (the program you have
  written) and {\tt person.java} (the automatically generated file)
in the same directory and compile them using the {\tt javac}  command. Add
the directories containing the API code, 
{\tt interprolog.jar}, and \texttt{log4j}  to the Java classpath:
%% 
\begin{itemize}
\item  \texttt{\textit{INSTALL\_DIR}/flora2/java/API/javaAPI/src} 
\item  \texttt{\textit{INSTALL\_DIR}/flora2/java/API/javaAPI/util}
\item  \texttt{\textit{INTERPROLOG\_DIR}/interprolog.jar}
\item  The LOG4J jar file, which might look like \texttt{log4j-1.2.12.jar}
  or similar. 
\end{itemize}
%% 


\item Run the code using the following command. For Unix-like systems
  (Linux, Mac, etc.), change {\tt
    \%VAR\%} to {\tt \$VAR} and replace $\backslash$ with {\tt /}.
\begin{verbatim}
%JAVA_BIN%\java -DXSB_ROOT_DIRECTORY=%XSB_ROOT_DIRECTORY%
                -DFLORA_DIRECTORY=%FLORA_DIRECTORY%
                -Djava.library.path=%XSB_ROOT_DIRECTORY%
                -DENGINE=%ENGINE%
                -classpath %CLASSPATH% flogicbasicsExample
\end{verbatim}

  The above command uses several shell variables, which are explained below.
  Instead of using the variables, one can substitute their values directly.

{\tt JAVA\_BIN}: This variable should point to the directory
containing the {\tt java}  and {\tt javac}  executables.

{\tt XSB\_ROOT\_DIRECTORY}: This variable should be set to the
directory containing the XSB executable.

{\tt FLORA\_DIRECTORY}: This variable should be set to the directory
containing the \FLSYSTEM system.

{\tt ENGINE}: This variable should be set to
{\tt Native} or  {\tt Subprocess}. The variable indicates whether InterProlog
should use the Java Native Interface (JNI) or sockets to communicate with
\FLSYSTEM.

{\tt CLASSPATH}: This variable should point to the directory
containing the API code (i.e., the correct path name of the
directory {\tt .../flora2/java/API})  and the file {\tt interprolog.jar}
(see below).

\end{itemize}

\section{Configuring the Java Interface}

The following applies \textbf{only} to the case when \FLSYSTEM was
downloaded from an SVN repository. Users of the prepackaged easy-to-install
\FLSYSTEM bundles should skip these steps.

The Java-to-\FLSYSTEM interface needs XSB, \FLSYSTEM, Java, and the
Interprolog library to be installed and configured. Here are the
configuration steps.

\begin{itemize}
\item Download J2SE 1.4.2 (or newer) from
  {\tt http://java.sun.com/j2se/},
and install it in some directory, say {\tt MY\_JAVA\_DIR}.
\item On Linux and other Unix-based systems,
  XSB needs to be configured in a special way
in order to interoperate with InterProlog. To build XSB with
InterProlog support, run the following commands in the {\tt build}
subdirectory of XSB.
The {\tt configure} script needs to be run only on Linux and
other Unix-based systems.
%%
\begin{verbatim}
./configure --with-interprolog
            --site-includes='MY_JAVA_DIR/includes MY_JAVA_DIR/includes/linux'
./makexsb
./makexsb dynmodule.
\end{verbatim}
%%
{\tt --site-includes} is needed only if Java is installed in a non-standard
place where {\tt configure} cannot find it automatically.

Under Cygwin, XSB also needs to be configured as above, but the last step,
{\tt ./makexsb dynmodule}, should be dropped.

\item Under Windows, XSB is distributed preconfigured for use with
  Interprolog. However, if you still need to rebuild it, use the following
  command sequence. First, you would need to configure XSB \emph{under
    Cygwin} (it is not possible to configure XSB under native Windows!) in a
  Cygwin terminal window.  
  %% 
\begin{verbatim}
./configure --with-interprolog
            --with-wind
            --site-includes='MY_JAVA_DIR\includes MY_JAVA_DIR\includes\win32'
\end{verbatim}
  %% 
  The path names in {\tt --site-include} must be Windows path names, not
  Cygwin's.

  Then, in a Windows (not Cygwin) command terminal, type:
  %% 
\begin{verbatim}
makexsb_wind  
\end{verbatim}
  %% 
  Note: this assumes that {\tt nmake} and the C++ compiler are available on
  your system. See the \FLSYSTEM manual, Section 1, for the instructions on
  obtaining these programs.
\item Rebuild \FLSYSTEM with the new XSB binary obtained in the
previous step.
\end{itemize}

\subsection{Building the Prepackaged Examples}

Sample applications of the Java-\FLSYSTEM interface
are found in the {\tt java/API/examples}  folder.
To build the code for the interface, use the scripts {\tt build.bat} or
{\tt build.sh} (or \texttt{build.bat} on Windows)
in the {\tt java/API}  folder.
To build the the examples, use the scripts
{\tt buildExample.sh} or  {\tt buildExample.bat} in the {\tt java/API/examples}
folder, whichever applies. For instance, to
build the {\tt flogicbasicsExample} example, use these commands on Linux,
Mac, and other Unix-like systems:
%%
\begin{verbatim}
    cd examples
    buildExample.sh flogicbasicsExample
\end{verbatim}
%%
On Windows, use this:
%%
\begin{verbatim}
    cd examples
    buildExample.bat flogicbasicsExample
\end{verbatim}
%%

To run the demos, use the scripts
{\tt runExample.sh} or  {\tt runExample.bat}  in the {\tt java/API/examples}
folder. For instance, to
run the {\tt flogicbasicsExample},  use this command on Linux, Mac, and the
like:
%%
\begin{verbatim}
    runExample.sh flogicbasicsExample
\end{verbatim}
%%
On Windows, use this:
%%
\begin{verbatim}
    runExample.bat flogicbasicsExample
\end{verbatim}
%%


%%% Local Variables: 
%%% mode: latex
%%% TeX-master: "flora-packages"
%%% End: 
